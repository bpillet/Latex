%LANGUAGE
\usepackage[francais]{babel}
%FONT
\usefonttheme[onlymath]{serif}
%SYMBOLS
\usepackage{bm,amsmath,amssymb,amsthm}
%GRAPHICS
\usepackage{graphicx}
%ENUMERATE
\usepackage{enumerate}
%LINKS
%\usepackage{hyperref} % toujours le mettre en dernier !
%INDEX
\usepackage{makeidx}
%COULEURS PERSOS
\usepackage{color}
%GESTION EPS
\usepackage{psfrag}
%FLECHES, SYMBOLES, GRAPHES, AUTOMATES
\usepackage{tikz}
\usetikzlibrary{matrix,arrows,automata}
%Remind to use "ampersand remplacement = \&" when using pgf matrix
%ou ajout de l'option [fragile] en début de frame

%ALGORITHMES
\usepackage[ruled,vlined,french]{algorithm2e}


\newenvironment{changemargin}[2]{%
\begin{list}{}{%
\setlength{\topsep}{0pt}%
\setlength{\leftmargin}{#1}%
\setlength{\rightmargin}{#2}%
\setlength{\listparindent}{\parindent}%
\setlength{\itemindent}{\parindent}%
\setlength{\parsep}{\parskip}%
%\setwidth{\topmargin}{-.1cm}%
}%
\item[]}{\end{list}}

%MATHBB\newcommand\N{\mathbb{N}} % Entiers naturels
\newcommand\Z{\mathbb{Z}} % Entiers relatifs
\newcommand\Q{\mathbb{Q}} % Rationnels
\newcommand\R{\mathbb{R}} % Réels
\newcommand\C{\mathbb{C}} % Complexes
\newcommand\Ha{\mathbb{H}} % Quaternions
\newcommand\K{\mathbb{K}} % Corps quelconque
\newcommand\E{\mathbb{E}} % Espérance
\newcommand\F{\mathbb{F}} % Corps fini
\newcommand\Pro{\mathbb{P}} %  Proba
\newcommand\B{\mathbb{B}} % aucune idée
\newcommand\T{\mathbb{T}} % Tore
%\newcommand\S{\mathbb{S}}
\def\S{\mathbb{S}} % Sphère
%\newcommand\k{\mathbb{k}}

%MATHCAL
\newcommand\Ll{\mathcal{L}}
\newcommand\Oo{\mathcal{O}}
\newcommand\Cc{\mathcal{C}}
\newcommand\Pp{\mathcal{P}}
\newcommand\Ff{\mathcal{F}}
\newcommand\Bb{\mathcal{B}}
\newcommand\Aa{\mathcal{A}}
\newcommand\Mm{\mathcal{M}}
\newcommand\Tt{\mathcal{T}}
\newcommand\Hh{\mathcal{H}}

\newcommand\Gl{\mathcal{G}\ell}

%MATHFRAC
\DeclareMathOperator\Sf{\mathfrak{S}}

%Category theory
%\DeclareMathOperator\domain{\text{domain}}
%\DeclareMathOperator\codom{\text{codomain}}
%\DeclareMathOperator\CC{\textbf{C}}
%\def\Hom{\text{Hom}}
%\DeclareMathOperator\0{\textbf{0}}
%\def\1{\textbf{1}}
%\DeclareMathOperator\2{\textbf{2}}
%\DeclareMathOperator\3{\textbf{3}}
%\DeclareMathOperator\Cat{\textbf{Cat}}

\DeclareMathOperator{\dd}{d\!} 	% d d'intégration comme dans dµ ou dt

\DeclareMathOperator{\demi}{\frac{1}{2}} % fraction 1/2

\DeclareMathOperator{\Vv}{| \! | \! |} % Norme subordonnée ||| A |||

\DeclareMathOperator{\sh}{sh} % sinus hyperbolique

\DeclareMathOperator{\Div}{div} % divergence

\DeclareMathOperator{\image}{Im} % Image

\DeclareMathOperator{\trans}{^\textbf{t}\!\!} %Tranposée à gauche

% Interval d'entiers ou interprétation d'un terme en logique.
\DeclareMathOperator{\llbracket}{[\![}
\DeclareMathOperator{\rrbracket}{]\!]}

\newcommand\name[2]{{#1}\textsc{ {#2}}} % utilisation : \name{Stefàn}{Banach}

\newcommand\ens[2]{\left\lbrace {#1} \; \left| \; {#2} \right. \right\rbrace } % utilisation : \ens{(x,y) \in \R^2}{x^2 + y^2 = 1}

\newcommand\Hom{\text{Hom}} % Hom-set

\newcommand\End{\text{End}} % Endomorphismes

\newcommand\ssi{\; \textbf{ssi} \;} % ssi

\newcommand\NLd[1]{\left\Vert {#1} \right\Vert_{L^2}} % Norme 2 : \NLd{f} = || f ||_2

\newcommand\dpp[2]{\dfrac{\partial{#1}}{\partial{#2}}} %\dpp{f}{x} = df/dx

\newcommand\quot[1]{\Z/{#1}\Z} % \quot{n} = Z/nZ

\newcommand\1[1]{\mathbf{1}_{\{#1\}}} % Fonction indicatrice

\newcommand\Card{\text{Card}} % Cardinal

%\newcommand\Legendre[2]{\left(\begin{array}{c}{#1} \\ {#2}\end{array}\right)}
\newcommand\Legendre[2]{\left(\dfrac{{#1}}{{#2}}\right)}  % Symbole de Legendre et Jacobi

%%%%%%%%%%%%%%%%%%%%%%%%%%%%%%%%%%%%%%%%%%%%%%%%%%%%%%%%%

\definecolor{DarkRed}{rgb}{0.55,0,0.1} 
\definecolor{DarkBlue}{rgb}{0.1,0,0.55}  
\definecolor{DarkGreen}{rgb}{0,0.55,0.1} 
%Pour utiliser sur le texte "banane" écrire {\color{DarkRed} banane}
%... peut s'utiliser sur de larges parties du texte.

%%%%%%%%%%%%%%%%%%%%%%%%%%%%%%%%%%%%%%%%%%%%%%%%%%%%%%%%%

\newtheorem{lem}{Lemme}
\newtheorem{rem}[theorem]{Remarque}
\newtheorem{thm}[theorem]{Théorème}
\newtheorem{cor}[theorem]{Corollaire}

\theoremstyle{definition}
\newtheorem{defi}{Définition}
\newtheorem{prop}{Proposition}

\newtheorem{proofs}{Démonstration (suite)}
\newtheorem{proofd}{Démonstration}


\AtBeginSubsection[] {
  \begin{frame}<beamer>{}
    \tableofcontents[currentsection,currentsubsection]
  \end{frame}
}

\AtBeginSection[] {
  \begin{frame}<beamer>{}
    \tableofcontents[currentsection,currentsubsection]
  \end{frame}
}
