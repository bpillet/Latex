\documentclass[11pt]{lettre}

\usepackage[utf8]{inputenc}
\usepackage[T1]{fontenc}
\usepackage[francais]{babel}

\begin{document}
\begin{letter}{À \textbf{Monsieur le recteur de l'académie de \textsc{Lille}}\\
						DPE\,	5ième bureau\\
						20 rue St Jacques\\
						59000 LILLE }
\address{Basile~\textsc{Pillet}\\
8 allée de Berne,\\
35200 RENNES}
\lieu{Rennes}
\name{Basile~\textsc{Pillet}}
\email{basile.pillet@univ-rennes1.fr}
\telephone{06 82 05 05 29}
\nofax
\def\concname{Objet :~} % On définit ici la commande 'objet'
\conc{Doctorant nouveau titulaire candidat aux fonctions d'ATER}
%\nref{Références de la lettre, de votre point de vue}
%\vref{Références de la lettre, du point de vue de votre interlocuteur} 
\opening{Monsieur le recteur,}

Je suis agrégé de mathématiques depuis 2012 et depuis le 1er septembre 2013, je suis doctorant contractuel au laboratoire IRMAR à Rennes. J'ai validé mon stage d'agrégation à travers mon monitorat de doctorant à l'université de Rennes 1.

À la suite du mouvement inter-académique de cette année, j'ai été affecté à l'académie de Lille.

Cependant l'an prochain (année scolaire 2016-2017) je souhaite poursuivre ma thèse pour une 4ième année, financée par un poste ATER. J'ai candidaté à un poste de titulaire à zone de remplacement au mouvement intra-académique comme conseillé.

Je vous fait donc par la présente, une demande de détachement pour la rentrée 2016 sous réserve que j'obtienne un poste d'ATER. Je m'engage alors à vous fournir tous les documents relatifs à ce poste aussi tôt que possible.

Dans l'hypothèse où le poste d'ATER ne me soit pas proposé, je sollicite une mise en disponibilité pour l'année scolaire 2016-2017 pour pouvoir tout de même terminer ma thèse.

\closing{Je vous prie de croire, Monsieur le recteur, en l’assurance de mes respectueuses salutations et en mon profond attachement au service public d’éducation.}

\end{letter}
\end{document} 