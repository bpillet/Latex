%CONSEILS DE SAN POUR LE COPIER-COLLER
%\usepackage{cmlgc}
\usepackage{ucs}
%\usepackage[utf8x]{inputenc}
%\usepackage[T1]{fontenc}
%GEOMETRY
\usepackage{geometry}
%\geometry{a4paper, hmargin=50pt, vmargin=75pt} % Régler les marges horizontales et verticales.
%LANGUAGE
\usepackage[english,francais]{babel}
%SYMBOLS
\usepackage{bm,amsmath,amssymb,amsthm}
%HEADINGS
%\pagestyle{headings} % Pour avoir des entêtes rappelant le titre de la section en cours sur chaque page.
%THEOREMS
\usepackage{thmbox} % Met les théorèmes dans des boites
%GRAPHICS
\usepackage{graphicx} % Permet de mettre des images
%ENUMERATE
\usepackage{enumerate} % Permet de faire des énumérations genre (i), (ii), (iii).
%INDEX
\usepackage{makeidx} % Permet de construire un index sémantique (comme à la fin des livres) trop la classe !
%COULEURS PERSOS
\usepackage{color}
%GESTION EPS
\usepackage{psfrag} % Aucune idée de ce à quoi ça sert.
%FLECHES, SYMBOLES, GRAPHES, AUTOMATES
\usepackage{tikz} % ajoute des symboles de flèches et autres.
\usetikzlibrary{matrix,arrows,automata} % Permet de constuire des graphes.
%ALGORITHMES
\usepackage[ruled,vlined,french]{algorithm2e} % Permet d'écrire des algos.

%% DIVERS
%\usepackage{multido}
%\usepackage{showlabels}

%LINKS
\usepackage{hyperref} % toujours le mettre en dernier !
\hypersetup{ colorlinks = true, linkcolor = black, urlcolor = blue, citecolor = blue }

%MATHBB
\newcommand\N{\mathbb{N}} % Entiers naturels
\newcommand\Z{\mathbb{Z}} % Entiers relatifs
\newcommand\Q{\mathbb{Q}} % Rationnels
\newcommand\R{\mathbb{R}} % Réels
\newcommand\C{\mathbb{C}} % Complexes
\newcommand\Ha{\mathbb{H}} % Quaternions (Ha__milton)
\newcommand\K{\mathbb{K}} % Corps quelconque
\newcommand\E{\mathbb{E}} % Espérance
\newcommand\F{\mathbb{F}} % Corps fini
\newcommand\Pro{\mathbb{P}} %  Proba
\newcommand\B{\mathbb{B}} % aucune idée
\newcommand\T{\mathbb{T}} % Tore
%\newcommand\S{\mathbb{S}}
\def\S{\mathbb{S}} % Sphère
%\newcommand\k{\mathbb{k}}

%MATHCAL
\newcommand\Oo{\mathcal{O}}
\newcommand\Uu{\mathcal{U}}
\newcommand\Gg{\mathcal{G}}
\newcommand\Ll{\mathcal{L}} % Applications linéaires 
\newcommand\Cc{\mathcal{C}} % Fonctions continues
\newcommand\Pp{\mathcal{P}} % Parties d'un ensemble
\newcommand\Ff{\mathcal{F}} % Tribu / Ensemble des fonctions …
\newcommand\Bb{\mathcal{B}} % Boréliens
\newcommand\Aa{\mathcal{A}} % Tribu
\newcommand\Mm{\mathcal{M}} % Matrices
\newcommand\Nn{\mathcal{N}} % …
\newcommand\Tt{\mathcal{T}} % Topologie
\newcommand\Hh{\mathcal{H}} % Espace de Hilbert

\newcommand\Gl{\mathcal{G}\ell} % Gl_n (K)

%MATHFRAC
\DeclareMathOperator\Sf{\mathfrak{S}} % Groupe symétrique
\DeclareMathOperator\Al{\mathfrak{A}} % Groupe alterné

%Category theory
%\DeclareMathOperator\domain{\text{domain}}
%\DeclareMathOperator\codom{\text{codomain}}
%\DeclareMathOperator\CC{\textbf{C}}
%\def\Hom{\text{Hom}}
%\DeclareMathOperator\0{\textbf{0}}
%\def\1{\textbf{1}}
%\DeclareMathOperator\2{\textbf{2}}
%\DeclareMathOperator\3{\textbf{3}}
%\DeclareMathOperator\Cat{\textbf{Cat}}

\DeclareMathOperator{\dd}{d\!} 	% d d'intégration comme dans dµ ou dt

\DeclareMathOperator{\demi}{\frac{1}{2}} % fraction 1/2

\DeclareMathOperator{\Vv}{| \! | \! |} % Norme subordonnée ||| A |||

\DeclareMathOperator{\sh}{sh} % sinus hyperbolique

\DeclareMathOperator{\Div}{div} % divergence

\DeclareMathOperator{\image}{Im} % Image

\DeclareMathOperator{\trans}{^\textbf{t}\!\!} %Tranposée à gauche

% Interval d'entiers ou interprétation d'un terme en logique.
\DeclareMathOperator{\llbracket}{[\![}
\DeclareMathOperator{\rrbracket}{]\!]}

\newcommand\name[2]{{#1}\textsc{ {#2}}} % utilisation : \name{Stefàn}{Banach}

\newcommand\ens[2]{\left\lbrace {#1} \; \left| \; {#2} \right. \right\rbrace } % utilisation : \ens{(x,y) \in \R^2}{x^2 + y^2 = 1}

\newcommand\Hom{\text{Hom}} % Hom-set

\newcommand\End{\text{End}} % Endomorphismes

\newcommand\ssi{\; \textbf{ssi} \;} % ssi

\newcommand\NLd[1]{\left\Vert {#1} \right\Vert_{L^2}} % Norme 2 : \NLd{f} = || f ||_2

\newcommand\dpp[2]{\dfrac{\partial{#1}}{\partial{#2}}} %\dpp{f}{x} = df/dx

\newcommand\quot[1]{\Z/{#1}\Z} % \quot{n} = Z/nZ

\newcommand\1[1]{\mathbf{1}_{\{#1\}}} % Fonction indicatrice
\newcommand\carac[1]{\mathbf{1}_{#1}} % Fonction indicatrice 2

\newcommand\Card{\text{Card}} % Cardinal

%\newcommand\Legendre[2]{\left(\begin{array}{c}{#1} \\ {#2}\end{array}\right)}
\newcommand\Legendre[2]{\left(\dfrac{{#1}}{{#2}}\right)}  % Symbole de Legendre et Jacobi

%%%%%%%%%%%%%%%%%%%%%%%%%%%%%%%%%%%%%%%%%%%%%%%%%%%%%%%%%

\definecolor{DarkRed}{rgb}{0.55,0,0.1} 
\definecolor{DarkBlue}{rgb}{0.1,0,0.55}  
\definecolor{DarkGreen}{rgb}{0,0.55,0.1} 
%Pour utiliser sur le texte "banane" écrire {\color{DarkRed} banane}
%... peut s'utiliser sur de larges parties du texte.

%%%%%%%%%%%%%%%%%%%%%%%%%%%%%%%%%%%%%%%%%%%%%%%%%%%%%%%%%

\newtheorem[M]{lem}{Lemme}
\newtheorem[S]{exemple}{Exemple}
\newtheorem[S]{exo}{Exercice}
\newtheorem[S]{rem}{Remarque}
\newtheorem[M,cut=false]{thm}{Théorème}

\theoremstyle{definition}
\newtheorem[M,cut=false]{defi}{Définition}
\newtheorem[S,cut=false]{prop}{Proposition}
\newtheorem[S]{cor}{Corollaire}
\newtheorem[S,cut=false, headstyle=\bfseries\boldmath{Exercise},bodystyle=\noindent]{Eno}{}