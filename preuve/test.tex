\documentclass[draft]{article}
\usepackage[T1]{fontenc}
\usepackage[utf8]{inputenc}
\usepackage{ucs}
\usepackage{bm,amsmath,amssymb}

\usepackage{preuve}

\DeclareMathOperator\R{\mathbb{R}} % Réels
\newcommand\Iff[2]{\left[{#1}, {#2}\right]}
\newcommand\Iof[2]{\left]{#1}, {#2}\right]}
\newcommand\Ifo[2]{\left[{#1}, {#2}\right[\:}
\newcommand\Ioo[2]{\left]{#1}, {#2}\right[\:}

\begin{document}

\section{Une première preuve}
\mq{$\forall x \in \R, \exists y \in \R, x > y$}
\soit{$x \in \mathbb R$}
\mq{$\exists y \in \R, x > y$}
\onpose{$y = x - 1$}
\mq{$x > y$}
On a bien $y \in \R$ et de plus comme $-1 < 0$, on a $x - 1 < x$ et donc $y < x$. Ce qui peut se réécrire $x > y$.
\finpose{$y$}
On a donc montré $\exists y \in \R, x>y$.
\finsoit{$x$}
Ainsi on a montré $\forall x \in \R, \exists y \in \R, x > y$. CQFD.

\section{Par l'absurde}
\mq{$0 \neq 1$}
\absurde{$0 = 1$}
On a donc $1 = 2$ ou encore par symétrie, $2 = 1$. Or le pape et moi somme deux personnes distinctes donc puisque $2 = 1$, le pape et moi somme une seule et même personne. Autrement dit : je suis le pape. C'est absurde car je ne parle pas latin et je n'ai même pas de chapeau.
\contradiction{$0 = 1$}
Ainsi on a montré que $0 \neq 1$.

\pagebreak

\section{Continuité de la fonction sinus}
\mq{$\forall \varepsilon > 0, \exists \eta > 0, \forall x \in \R, \left(\vert x \vert < \eta\right) \Rightarrow \left(\vert \sin(x) \vert < \varepsilon \right)$}

\soit{$\varepsilon > 0$}

\mq{$\exists \eta > 0, \forall x \in \R, \left(\vert x \vert < \eta\right) \Rightarrow \left(\vert \sin(x) \vert < \varepsilon \right)$}

\onpose{$\eta = \varepsilon$}

On a $\eta > 0$ par définition de $\varepsilon$.

\mq{$\forall x \in \R, \left(\vert x \vert < \eta\right) \Rightarrow \left(\vert \sin(x) \vert < \varepsilon \right)$}

\soit{$x \in \R$}

\mq{$\left(\vert x \vert < \eta\right) \Rightarrow \left(\vert \sin(x) \vert < \varepsilon \right)$}

\onsuppose{$\vert x \vert < \eta$}

\mq{$\vert \sin(x) \vert < \varepsilon$}

Par inégalité des accroissement finis appliqué à la fonction $\mathcal C^1$ sinus sur $[0,x]$ (ou $[x,0]$ suivant le signe de $x$), on a $\vert \sin(x) - \sin(0)\vert \leq 1 \times \vert x - 0\vert$ ou encore $\vert \sin(x) \vert \leq \vert x \vert$.

Dès lors, comme $\vert x \vert < \eta$ et $\eta = \varepsilon$ on a $\vert \sin(x) \vert < \varepsilon$.

\finsuppose{$\vert \sin(x) \vert < \varepsilon$}{$\vert x \vert < \eta$}
Ainsi on a montré $\left(\vert x \vert < \eta\right) \Rightarrow \left(\vert \sin(x) \vert < \varepsilon \right)$

\finsoit{$x$}
Donc on a montré $\forall x \in \R, \left(\vert x \vert < \eta\right) \Rightarrow \left(\vert \sin(x) \vert < \varepsilon \right)$.

\finpose{$\eta$}

Donc on a montré $\exists \eta > 0, \forall x \in \R, \left(\vert x \vert < \eta\right) \Rightarrow \left(\vert \sin(x) \vert < \varepsilon \right)$.

\finsoit{$\varepsilon$}
Et finalement on a montré que la fonction $\sin$ est continue en $0$ (en utilisant qu'elle est de classe $\mathcal C^1$, mais là n'est pas la question).


\pagebreak

\section{Disjonction de cas}

\subsection{Connexité des intervalles}
Soient $a,b \in \R$ avec $a < b$
\mq{l'intervalle $\Ifo a b$ est connexe}
c'est-à-dire : 
\mq{$\forall u, v \in \Ifo a b, \Iff u v \subseteq \Ifo a b$}

\soit{$u, v \in \Ifo a b$}
\mq{$\Iff u v \subseteq \Ifo a b$}
\begin{disjonction}
  \cas{$u > v$}
  Dans ce cas $\Iff u v$ est vide et l'inclusion est automatique.
  \fincas

  \cas{$u \leq v$}
  
  \mq{$\forall x \in \Iff u v, x \in \Ifo a b$}
  \soit{$x \in \Iff u v$}
  \mq{$x \in \Ifo a b$}
  On a $u \leq x \leq v$ par définition de $x$, et de plus
  \begin{itemize}
  \item comme $u \in \Ifo a b$, alors $a \leq u$, et ainsi $a \leq u \leq x$
  \item et de même comme $v \in \Ifo a b$, alors $v <b$ et donc $x \leq v < b$.
  \end{itemize}
  En conclusion, par transitivité $a \leq x < b$ ce qui signifie que $x \in \Ifo a b$.
  \finsoit{$x$}
  On a donc montré que $\Iff u v \subseteq \Ifo a b$ dans ce cas.
  \fincas
\end{disjonction}
L'inclusion étant vraie dans tous les cas, on a montré $\Iff u v \subseteq \Ifo a b$
\finsoit{$u,v$}
On a donc montré que $\Ifo a b$ est connexe.

\pagebreak

\subsection{Résolution d'une inéquation}

Soit à résoudre $\vert x - 3 \vert \leq \vert 2x - 3 \vert$

\subsubsection{Analyse}
\soit{$x\in \R$ solution de l'inéquation}

\begin{disjonction}[]%
  \cas{$x - 3 < 0$}
  Alors $\vert x - 3 \vert = - x + 3$.
  \begin{disjonction}[Sous-cas]
    \cas{$2x - 3 \geq 0$}
    On a alors $\vert 2x - 3 \vert = 2x - 3$ et donc $x$ vérifie $3 - x \leq 2x - 3$ ou encore $3x \geq 6$ c'est-à-dire $x \geq 2$. Or on a supposé $x < 3$ donc $x \in \Ifo 2 3$.
    \fincas

    \cas{$2x - 3 < 0$}
    On a alors $\vert 2x - 3 \vert = -2x + 3$ et donc $x$ vérifie $3 - x \leq 3 - 2x$ ou encore $x \leq 0$.
    \fincas
  \end{disjonction}
  On peut donc avoir les sous-cas suivants :  $x \in \Ifo 2 3$ ou $x \leq 0$.
  
  \fincas
  
  \cas{$x - 3 \geq 0$}
  Alors $\vert x - 3 \vert = x - 3$ et d'autre part $x \geq 3$ et donc $2x \geq 6$ ainsi $2x - 3 \geq 0$. On a donc $\vert 2x - 3 \vert = 2x - 3$. Il s'en suit donc que $x$ vérifie $x - 3 \leq 2x - 3$ ou encore $x \geq 0$. Ce qui est automatique sous notre hypothèse.
  \fincas

\end{disjonction}
En conclusion on peut avoir les cas suivants : $x \leq 0$, $x \in \Ifo 2 3$ et $x \geq 3$.

\finsoit{$x$} Ainsi toute solution de l'équation est dans $\Iof {-\infty} 0 \cup \Ifo 2 {+\infty}$.
\end{document}

%%% Local Variables:
%%% mode: latex
%%% End:
